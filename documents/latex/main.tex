% ...existing code...
\documentclass[12pt]{article}
\usepackage[utf8]{inputenc}
\usepackage[T1]{fontenc}
\usepackage[brazil]{babel}
\usepackage{lmodern}

\title{Segurança da Informação}
\author{Alexandre Neves, Felipe Fidelis}
\date{\today}

\begin{document}
\maketitle

\section{Introdução}
Este é um documento mínimo em LaTeX. Substitua o conteúdo conforme necessário.

\section{Política de Segurança e Uso dos Computadores dos Laboratórios}

\subsection{Objetivo}

Estabelecer diretrizes e responsabilidades para garantir a segurança, integridade, disponibilidade e uso adequado dos recursos de informática (hardware, software e rede) presentes nos laboratórios do Instituto Federal Goiano - Campus Ceres, prevenindo acessos não autorizados, danos e mau uso.

\subsection{Âmbito de Aplicação}

Esta política aplica-se a todos os usuários (estudantes, professores, pesquisadores e técnicos) que utilizam os computadores localizados nos laboratórios da instituição.

\subsection{Diretrizes Gerais de Uso (Para Todos os Usuários)}

\subsubsection{Uso de Contas de Acesso}
\begin{enumerate}
    \item \textbf{Contas Individuais e Não Compartilháveis:} Cada usuário (aluno e professor) deve utilizar sua conta de acesso pessoal e intransferível. \textbf{É estritamente proibido compartilhar senhas ou utilizar contas de terceiros.}
    \item \textbf{Senhas Fortes:} As senhas devem seguir os requisitos mínimos de complexidade definidos pela instituição (ex: mínimo de 8 caracteres, com letras maiúsculas, minúsculas, números e símbolos).
    \item \textbf{Logout e Bloqueio:} Os usuários devem sempre efetuar \emph{logout} ou bloquear a estação de trabalho ao se ausentarem, mesmo que por um breve período.
    \item \textbf{Monitoramento:} A instituição reserva-se o direito de monitorar o uso dos equipamentos para fins de manutenção e segurança, conforme a legislação vigente.
\end{enumerate}

\subsection{Software e Configurações}
\begin{enumerate}
    \item \textbf{Instalação e Configuração de Ambiente de Desenvolvimento (Sistemas Multiusuários):} É proibida a modificação ou remoção de qualquer software, aplicativo ou arquivo executável que faça parte do sistema operacional base. \textbf{EXCEÇÃO - Configuração de Ambiente:} Nos laboratórios dos cursos de Sistemas de Informação, Informática para Internet e Inteligência Artificial, é permitido que o aluno realize a configuração do seu ambiente de desenvolvimento pessoal (\emph{personal environment}) para fins curriculares, \textbf{desde que}:
    \begin{enumerate}
        \item A instalação de bibliotecas, pacotes e \emph{frameworks} seja feita utilizando \textbf{gerenciadores de pacotes com escopo local} (ex: \texttt{pip --user}, \texttt{npm} em modo local, gerenciadores de ambientes virtuais como \texttt{conda} ou \texttt{venv}).
        \item A configuração de variáveis de ambiente, como o \texttt{PATH}, seja feita estritamente através dos arquivos de configuração da \emph{shell} no diretório \texttt{home} do usuário.
        \item \textbf{Não seja utilizada a elevação de privilégios de administrador} (\texttt{sudo} ou \texttt{root}) para qualquer instalação ou configuração.
    \end{enumerate}
    
    \item \textbf{Uso de Contêineres e Virtualização:} Para tarefas que exijam contêineres (\emph{Docker, Podman}), o aluno deve, prioritariamente, utilizar soluções \emph{rootless} (que não requerem acesso de administrador) ou ambientes de desenvolvimento pré-configurados e aprovados pela TI. A concessão de acesso ao grupo \texttt{docker} é proibida por representar risco de segurança à máquina hospedeira.
    
    \item \textbf{Downloads Não Autorizados:} É proibido o download e/ou armazenamento de conteúdo ilegal, malicioso (\emph{vírus, malware}), pornográfico ou que viole direitos autorais.
    \item \textbf{Alteração de Configurações:} É proibida a alteração de configurações de sistema, rede, papel de parede, \emph{screensaver} ou qualquer ajuste que comprometa o padrão operacional da máquina.
\end{enumerate}

\subsubsection{Uso da Rede e Internet}
\begin{enumerate}
    \item \textbf{Acesso Remoto (\emph{SSH, VNC, RDP}):} O acesso remoto entre estações de alunos é estritamente proibido. O uso de protocolos de acesso remoto para fins acadêmicos ou de pesquisa deve ser formalmente solicitado e limitado a servidores específicos da instituição, conforme as regras de \emph{firewall} e segurança.
    \item \textbf{Comportamento Ético:} É proibido utilizar a rede para fins que violem a lei, promovam \emph{hacking, phishing} ou qualquer atividade que cause prejuízo à instituição ou a terceiros.
\end{enumerate}

\subsection{Diretrizes Específicas para Alunos}
\begin{enumerate}
    \item \textbf{Uso Exclusivo para Fins Acadêmicos:} Os computadores dos laboratórios destinam-se primariamente a atividades de ensino, pesquisa e extensão. O uso pessoal excessivo (jogos, redes sociais, \emph{streaming}) pode ser restringido.
    \item \textbf{Armazenamento Temporário:} Arquivos pessoais devem ser salvos em serviços de armazenamento em nuvem da instituição (se disponíveis) ou em mídias externas. A instituição não se responsabiliza por arquivos salvos no disco local, que podem ser apagados a qualquer momento (ex: no \emph{reboot} da máquina).
\end{enumerate}

\subsection{Diretrizes Específicas para Professores/Docentes}

\subsubsection{Segurança da Estação Docente (Mesa do Professor)}
\begin{enumerate}
    \item \textbf{Desativação de Serviços Desnecessários:} O serviço \textbf{SSH (ou qualquer outro serviço de acesso remoto como VNC ou RDP)} deve ser \textbf{desativado} na estação do professor por padrão. Ele só poderá ser ativado temporariamente para propósitos didáticos específicos, e deve ser desativado imediatamente após o uso.
    \item \textbf{Firewall Rigoroso:} O \emph{firewall} da estação docente deve estar sempre ativo e configurado para \textbf{bloquear todas as conexões de entrada}, exceto aquelas absolutamente necessárias para o funcionamento em sala de aula (ex: projeção de tela).
    \item \textbf{Contas de Usuário:} O professor deve utilizar uma \textbf{conta de usuário padrão (não administrador)} para as aulas, reservando a conta de administrador para tarefas de manutenção ou instalação de software, se necessário.
    \item \textbf{Autenticação Dupla (Se Possível):} Em máquinas com acesso a sistemas sensíveis, considerar o uso de autenticação de dois fatores ou o bloqueio por senha robusta no \emph{login} inicial.
    \item \textbf{Acesso Físico:} O professor deve garantir que a estação docente esteja fisicamente segura (ex: \emph{case} com trava ou cabo de segurança), limitando o acesso a portas USB ou físicas por alunos.
\end{enumerate}

\subsubsection{Responsabilidades do Professor em Sala de Aula}
\begin{enumerate}
    \item \textbf{Conscientização:} O professor deve orientar os alunos sobre esta política no início de cada disciplina.
    \item \textbf{Monitoramento:} O professor é o responsável imediato por monitorar o comportamento dos alunos no laboratório e reportar atividades suspeitas ou violações de segurança ao setor de TI.
    \item \textbf{Projeção de Tela:} Antes de iniciar a projeção (\emph{datashow}), o professor deve verificar a tela, garantindo que nenhuma aplicação não autorizada esteja sendo executada.
\end{enumerate}

\subsection{Medidas Disciplinares}

O não cumprimento desta Política de Segurança e Uso constitui uma violação das normas internas da instituição e acarretará as seguintes medidas disciplinares:
\begin{enumerate}
    \item \textbf{Advertência:} Em casos de primeira ocorrência e infração leve.
    \item \textbf{Suspensão de Acesso:} Suspensão temporária do acesso aos laboratórios e/ou à rede institucional.
    \item \textbf{Processo Disciplinar:} Em casos de infrações graves, reincidência ou prejuízos à instituição, o usuário será submetido a um processo disciplinar, podendo resultar em expulsão (para alunos) ou outras medidas cabíveis.
    \item \textbf{Ações Legais:} A instituição poderá tomar medidas legais em casos de crimes cibernéticos ou danos materiais e morais, conforme a legislação brasileira.
\end{enumerate}

\subsection{Revisão da Política}

Esta política será revisada e atualizada anualmente ou sempre que houver mudanças significativas na infraestrutura tecnológica ou nas necessidades de segurança da instituição.

\section{Implementação de Hardening SSH}

\subsection{Objetivo}

Documentar as técnicas de proteção (\emph{hardening}) aplicadas ao serviço SSH para mitigar ataques de força bruta, acesso não autorizado e outras vulnerabilidades comuns em ambientes de rede.

\subsection{Ambiente de Laboratório}

O laboratório virtual consiste em três máquinas virtuais configuradas com Vagrant:
\begin{itemize}
    \item \textbf{alvo} (192.168.56.10): Máquina \textbf{sem proteções} de segurança, representando um sistema vulnerável
    \item \textbf{alvo-hardened} (192.168.56.11): Máquina \textbf{com proteções} aplicadas, demonstrando boas práticas de segurança
    \item \textbf{atacante} (192.168.56.20): Máquina utilizada para simular ataques e testes de penetração
\end{itemize}

\subsection{Configurações de Hardening SSH Implementadas}

\subsubsection{1. Desabilitar Login Root via SSH}
\begin{verbatim}
PermitRootLogin no
\end{verbatim}

\textbf{Justificativa:} Impede que atacantes tentem login direto como usuário \texttt{root}, forçando-os a comprometer primeiro uma conta de usuário normal e depois escalar privilégios. Reduz drasticamente a superfície de ataque.

\textbf{Impacto em Ataques:}
\begin{itemize}
    \item \textbf{Sem proteção:} Atacante pode tentar diretamente \texttt{ssh root@IP}
    \item \textbf{Com proteção:} Mesmo descobrindo a senha do root, acesso SSH é negado
\end{itemize}

\subsubsection{2. Limitar Tentativas de Autenticação}
\begin{verbatim}
MaxAuthTries 3
\end{verbatim}

\textbf{Justificativa:} Reduz o número de tentativas de senha por conexão SSH de 6 (padrão) para 3, diminuindo a eficácia de ataques de força bruta automatizados.

\textbf{Impacto em Ataques:}
\begin{itemize}
    \item \textbf{Sem proteção:} 6 tentativas por conexão
    \item \textbf{Com proteção:} Apenas 3 tentativas; atacante precisa reconectar mais frequentemente
\end{itemize}

\subsubsection{3. Timeout de Login Reduzido}
\begin{verbatim}
LoginGraceTime 30
\end{verbatim}

\textbf{Justificativa:} Reduz o tempo máximo para completar autenticação de 120 segundos (padrão) para 30 segundos. Conexões lentas ou suspeitas são encerradas rapidamente.

\textbf{Impacto em Ataques:}
\begin{itemize}
    \item \textbf{Sem proteção:} Atacantes podem manter conexões abertas por 2 minutos
    \item \textbf{Com proteção:} Conexões inativas/lentas são fechadas em 30s
\end{itemize}

\subsubsection{4. Proibir Senhas Vazias}
\begin{verbatim}
PermitEmptyPasswords no
\end{verbatim}

\textbf{Justificativa:} Impede login em contas sem senha definida, eliminando um vetor de ataque óbvio.

\subsubsection{5. Fail2ban - Proteção Contra Força Bruta}

Configuração implementada em \texttt{/etc/fail2ban/jail.local}:
\begin{verbatim}
[sshd]
enabled = true
port = 22
maxretry = 3
bantime = 3600
findtime = 600
\end{verbatim}

\textbf{Funcionamento:}
\begin{itemize}
    \item Monitora log de autenticação: \texttt{/var/log/auth.log}
    \item Após \textbf{3 tentativas falhadas} em \textbf{10 minutos} (findtime)
    \item Bane o endereço IP por \textbf{1 hora} (bantime)
\end{itemize}

\textbf{Impacto em Ataques:}
\begin{itemize}
    \item \textbf{Sem proteção:} Atacante pode realizar milhares de tentativas sem restrição
    \item \textbf{Com proteção:} Após 3 falhas, IP bloqueado por 1 hora
    \item Ataques de força bruta SSH tornam-se \textbf{impraticáveis}
    \item Atacante precisaria de múltiplos IPs ou esperar 1h entre tentativas
\end{itemize}

\subsubsection{6. Firewall UFW (Uncomplicated Firewall)}
\begin{verbatim}
ufw default deny incoming
ufw default allow outgoing
ufw allow 22/tcp
ufw --force enable
\end{verbatim}

\textbf{Justificativa:} Implementa política de \emph{whitelist} - bloqueia todas as conexões de entrada por padrão, permitindo apenas SSH (porta 22).

\textbf{Impacto em Ataques:}
\begin{itemize}
    \item \textbf{Sem proteção:} Todas as portas acessíveis para varredura e exploração
    \item \textbf{Com proteção:} Apenas porta SSH visível; superfície de ataque drasticamente reduzida
\end{itemize}

\subsection{Comparação: Sistema Vulnerável vs Protegido}

\begin{table}[h]
\centering
\begin{tabular}{|l|l|l|}
\hline
\textbf{Proteção} & \textbf{Alvo (Vulnerável)} & \textbf{Alvo-Hardened} \\ \hline
Login Root SSH & Permitido & Bloqueado \\ \hline
Tentativas/conexão & 6 & 3 \\ \hline
Timeout login & 120s & 30s \\ \hline
Senhas vazias & Possível & Bloqueado \\ \hline
Ban após falhas & Nunca & 3 tentativas = 1h \\ \hline
Firewall & Inexistente & Ativo (só SSH) \\ \hline
Varredura de portas & Todas visíveis & Apenas SSH \\ \hline
\end{tabular}
\caption{Comparação de Configurações de Segurança}
\end{table}

\subsection{Cenário de Ataque: SSH Brute-Force}

\subsubsection{Contra Sistema Vulnerável (alvo)}
\begin{enumerate}
    \item Atacante executa: \texttt{ssh vagrant@192.168.56.10}
    \item Senha incorreta? Tenta novamente sem restrição
    \item Pode realizar 1000+ tentativas sem consequências
    \item \textbf{Resultado:} Eventualmente obtém acesso com senha correta
\end{enumerate}

\subsubsection{Contra Sistema Protegido (alvo-hardened)}
\begin{enumerate}
    \item Atacante executa: \texttt{ssh vagrant@192.168.56.11}
    \item Tentativa 1: senha incorreta
    \item Tentativa 2: senha incorreta
    \item Tentativa 3: senha incorreta
    \item \textbf{Fail2ban detecta} e bane IP 192.168.56.20 por 1 hora
    \item \textbf{Resultado:} Ataque bloqueado; conexões futuras recusadas
\end{enumerate}

\subsection{Boas Práticas Adicionais Recomendadas}

Além das configurações implementadas, recomenda-se:
\begin{enumerate}
    \item \textbf{Autenticação por chave SSH} ao invés de senha (\texttt{PasswordAuthentication no})
    \item \textbf{Mudar porta padrão SSH} de 22 para porta alta não-padrão
    \item \textbf{Implementar Two-Factor Authentication (2FA)} para SSH
    \item \textbf{Limitar usuários SSH} via \texttt{AllowUsers} ou \texttt{AllowGroups}
    \item \textbf{Log centralizado} para análise forense em servidor remoto
    \item \textbf{IDS/IPS} (Intrusion Detection/Prevention System) como Snort ou Suricata
\end{enumerate}

\subsection{Comandos para Aplicar Configurações}

Para aplicar o hardening SSH na máquina virtual:
\begin{verbatim}
vagrant provision alvo-hardened
\end{verbatim}

Para verificar status das proteções:
\begin{verbatim}
# Status do Firewall
sudo ufw status verbose

# Status do Fail2ban
sudo systemctl status fail2ban
sudo fail2ban-client status sshd

# Verificar configuração SSH
sudo grep -E "PermitRootLogin|MaxAuthTries|LoginGraceTime" \
  /etc/ssh/sshd_config
\end{verbatim}

\section{Testes de Validação das Proteções}

\subsection{Objetivo dos Testes}

Validar a eficácia das configurações de hardening SSH implementadas através de testes práticos que demonstrem as diferenças entre um sistema vulnerável e um sistema protegido.

\subsection{Ambiente de Teste}

Todos os testes foram executados a partir da VM \textbf{atacante} (192.168.56.20) contra as duas máquinas-alvo:
\begin{itemize}
    \item \textbf{alvo} (192.168.56.10): Sistema vulnerável sem proteções
    \item \textbf{alvo-hardened} (192.168.56.11): Sistema com hardening aplicado
\end{itemize}

\subsection{Teste 1: Varredura de Portas (Port Scanning)}

\subsubsection{Metodologia}
Utilização da ferramenta \texttt{nmap} para identificar portas abertas e serviços expostos em ambas as máquinas.

\subsubsection{Comandos Executados}
\begin{verbatim}
# Varredura simples (portas 1-100)
nmap -p 1-100 192.168.56.10
nmap -p 1-100 192.168.56.11

# Varredura sem ping (para firewall)
nmap -Pn -p 20-25 192.168.56.10
nmap -Pn -p 20-25 192.168.56.11
\end{verbatim}

\subsubsection{Resultados Obtidos}

\textbf{VM alvo (vulnerável):}
\begin{verbatim}
PORT   STATE  SERVICE
22/tcp open   ssh
Scan time: 13.12 segundos
\end{verbatim}

\textbf{VM alvo-hardened (protegida):}
\begin{verbatim}
Note: Host seems down (primeira tentativa)

Com -Pn:
PORT   STATE    SERVICE
20/tcp filtered ftp-data
21/tcp filtered ftp
22/tcp open     ssh
23/tcp filtered telnet
24/tcp filtered priv-mail
25/tcp filtered smtp
Scan time: 14.33 segundos
\end{verbatim}

\subsubsection{Análise dos Resultados}
\begin{itemize}
    \item A VM protegida bloqueia pacotes ICMP (ping), dificultando detecção inicial
    \item Firewall UFW filtra portas não autorizadas (status \texttt{filtered})
    \item Apenas porta 22 (SSH) permanece acessível no sistema protegido
    \item Tempo de scan ligeiramente maior devido ao timeout de portas filtradas
\end{itemize}

\subsection{Teste 2: Verificação de Status de Segurança}

\subsubsection{Metodologia}
Comparação direta do status dos componentes de segurança em ambas as VMs.

\subsubsection{Comandos Executados}
\begin{verbatim}
# Verificar firewall
vagrant ssh alvo -c "sudo ufw status"
vagrant ssh alvo-hardened -c "sudo ufw status"

# Verificar fail2ban
vagrant ssh alvo -c "sudo systemctl status fail2ban"
vagrant ssh alvo-hardened -c "sudo fail2ban-client status sshd"
\end{verbatim}

\subsubsection{Resultados Obtidos}

\begin{table}[h]
\centering
\begin{tabular}{|l|l|l|}
\hline
\textbf{Componente} & \textbf{Alvo (Vulnerável)} & \textbf{Alvo-Hardened (Protegido)} \\ \hline
Firewall UFW & Status: inactive & Status: active \\ \hline
Regras Firewall & Nenhuma & 22/tcp ALLOW Anywhere \\ \hline
Fail2ban & Não instalado & Active (running) \\ \hline
Monitoramento & Nenhum & /var/log/auth.log \\ \hline
IPs banidos & N/A & 0 (nenhum ataque ainda) \\ \hline
\end{tabular}
\caption{Comparação de Status dos Componentes de Segurança}
\end{table}

\subsection{Teste 3: Configurações SSH}

\subsubsection{Metodologia}
Inspeção direta do arquivo de configuração SSH (\texttt{/etc/ssh/sshd\_config}) em ambas as máquinas.

\subsubsection{Comandos Executados}
\begin{verbatim}
vagrant ssh alvo -c "sudo grep -E 'PermitRootLogin|MaxAuthTries|\
LoginGraceTime|PermitEmptyPasswords' /etc/ssh/sshd_config | grep -v '^#'"

vagrant ssh alvo-hardened -c "sudo grep -E 'PermitRootLogin|\
MaxAuthTries|LoginGraceTime|PermitEmptyPasswords' \
/etc/ssh/sshd_config | grep -v '^#'"
\end{verbatim}

\subsubsection{Resultados Obtidos}

\textbf{VM alvo (vulnerável):}
\begin{verbatim}
(Nenhuma configuração explícita - valores padrão)
\end{verbatim}

\textbf{VM alvo-hardened (protegida):}
\begin{verbatim}
LoginGraceTime 30
PermitRootLogin no
MaxAuthTries 3
PermitEmptyPasswords no
\end{verbatim}

\subsection{Teste 4: Enumeração de Usuários (User Enumeration)}

\subsubsection{Metodologia}
Ataque que tenta descobrir quais contas de usuário existem no sistema através de tentativas de conexão SSH. Atacantes usam essa técnica para identificar alvos válidos antes de realizar ataques de força bruta.

\subsubsection{Técnica Utilizada}
O ataque utiliza o método de tentar autenticação sem credenciais (\texttt{PreferredAuthentications=none}). Usuários válidos retornam mensagem específica "Permission denied", enquanto usuários inválidos podem ter comportamento diferente (timeout, mensagens distintas).

\subsubsection{Comandos Executados}
\begin{verbatim}
# Script executado da VM atacante
bash /vagrant/ataque_enumeracao_usuarios.sh 192.168.56.10
bash /vagrant/ataque_enumeracao_usuarios.sh 192.168.56.11

# Técnica base:
ssh -o PreferredAuthentications=none \
    -o StrictHostKeyChecking=no \
    -o ConnectTimeout=3 \
    usuario@IP 2>&1 | grep "Permission denied"
\end{verbatim}

\subsubsection{Lista de Usuários Testados}
root, admin, administrator, vagrant, ubuntu, guest, test, user, operator, backup, mysql, postgres, apache, nginx, www-data, nobody (total: 16 usuários comuns)

\subsubsection{Resultados Obtidos}

\textbf{VM alvo (vulnerável):}
\begin{verbatim}
Usuários CONFIRMADOS (16):
- root, admin, administrator
- vagrant, ubuntu, guest
- test, user, operator, backup
- mysql, postgres, apache, nginx
- www-data, nobody

Taxa de sucesso: 100% (16/16)
\end{verbatim}

\textbf{VM alvo-hardened (protegida):}
\begin{verbatim}
Usuários CONFIRMADOS (6):
- root, admin, administrator
- vagrant, ubuntu, guest

NÃO CONFIRMADOS (10):
- test, user, operator, backup
- mysql, postgres, apache, nginx
- www-data, nobody

Taxa de sucesso: 37.5% (6/16)
Timeouts: 62.5% das tentativas
\end{verbatim}

\subsubsection{Análise Comparativa}

\begin{table}[h]
\centering
\begin{tabular}{|l|c|c|}
\hline
\textbf{Métrica} & \textbf{Alvo (Vulnerável)} & \textbf{Alvo-Hardened} \\ \hline
Usuários Identificados & 16 & 6 \\ \hline
Taxa de Sucesso Ataque & 100\% & 37.5\% \\ \hline
Informação Vazada & Total & Parcial \\ \hline
Tempo Médio/Teste & ~3s & ~5s (timeouts) \\ \hline
Eficácia da Proteção & N/A & 62.5\% redução \\ \hline
\end{tabular}
\caption{Comparação de Enumeração de Usuários}
\end{table}

\subsubsection{Impacto de Segurança}

\textbf{Sistema Vulnerável:}
\begin{itemize}
    \item Atacante obtém lista completa de 16 usuários válidos
    \item Pode direcionar ataques de força bruta para contas confirmadas
    \item Identifica contas de serviço (mysql, apache, nginx) revelando serviços instalados
    \item Respostas rápidas facilitam enumeração automatizada
\end{itemize}

\textbf{Sistema Protegido:}
\begin{itemize}
    \item \textbf{LoginGraceTime 30s:} Conexões lentas/suspeitas são encerradas rapidamente
    \item \textbf{Fail2ban:} Múltiplas tentativas de conexão são detectadas e o IP pode ser banido
    \item \textbf{Timeouts:} 10 usuários não puderam ser confirmados (62.5\% menos informação)
    \item \textbf{Dificulta Automação:} Tempos de resposta variáveis dificultam scripts automatizados
\end{itemize}

\textbf{Vulnerabilidades Expostas no Sistema Vulnerável:}
\begin{enumerate}
    \item Revela presença de bancos de dados (mysql, postgres)
    \item Identifica servidor web ativo (apache, nginx, www-data)
    \item Mostra contas de backup potencialmente com privilégios elevados
    \item Fornece lista de alvos para próxima fase do ataque (força bruta)
\end{enumerate}

\subsection{Resultados Consolidados}

\begin{table}[h]
\centering
\begin{tabular}{|l|c|c|c|}
\hline
\textbf{Teste} & \textbf{Alvo} & \textbf{Alvo-Hardened} & \textbf{Status} \\ \hline
Port Scanning & Todas visíveis & Apenas SSH & \checkmark \\ \hline
Firewall UFW & Inativo & Ativo & \checkmark \\ \hline
Fail2ban & Não instalado & Ativo & \checkmark \\ \hline
SSH Hardening & Padrão & Configurado & \checkmark \\ \hline
Port Filtering & Nenhum & 5 portas filtradas & \checkmark \\ \hline
Enumeração Usuários & 16 encontrados & 6 encontrados & \checkmark \\ \hline
Redução Info Vazada & 0\% & 62.5\% & \checkmark \\ \hline
\end{tabular}
\caption{Resumo dos Resultados dos Testes}
\end{table}

\subsection{Conclusões dos Testes}

\subsubsection{Eficácia das Proteções}
\begin{enumerate}
    \item \textbf{Firewall UFW:} Funcionando corretamente, bloqueando todas as portas exceto SSH
    \item \textbf{Fail2ban:} Ativo e monitorando tentativas de acesso via SSH
    \item \textbf{SSH Hardening:} Todas as 4 configurações aplicadas com sucesso
    \item \textbf{Redução de Superfície de Ataque:} Apenas porta 22 acessível vs múltiplas portas potencialmente abertas
\end{enumerate}

\subsubsection{Impacto na Segurança}
A implementação das proteções resultou em:
\begin{itemize}
    \item \textbf{Visibilidade Reduzida:} Sistema protegido não responde a pings, dificultando detecção
    \item \textbf{Filtragem de Portas:} 5 de 6 portas testadas retornam status "filtered" ao invés de "closed"
    \item \textbf{Configuração Robusta:} SSH configurado com limites de tentativas e timeouts reduzidos
    \item \textbf{Monitoramento Ativo:} Fail2ban pronto para banir IPs após 3 tentativas falhadas
\end{itemize}

\subsubsection{Vulnerabilidades Mitigadas}
\begin{itemize}
    \item Ataques de força bruta SSH (fail2ban + MaxAuthTries)
    \item Exploração de portas abertas desnecessárias (UFW)
    \item Login direto como root (PermitRootLogin no)
    \item Conexões SSH prolongadas (LoginGraceTime 30)
    \item Contas sem senha (PermitEmptyPasswords no)
    \item \textbf{Enumeração de usuários} (LoginGraceTime + fail2ban reduzem sucesso em 62.5\%)
\end{itemize}

\subsection{Recomendações Finais}

Com base nos testes realizados, confirma-se que o hardening SSH foi implementado com sucesso. Para ambientes de produção, recomenda-se adicionalmente:
\begin{enumerate}
    \item Implementar autenticação por chave SSH ao invés de senha
    \item Configurar logging centralizado para análise forense
    \item Realizar testes de penetração periódicos
    \item Manter sistema operacional e serviços sempre atualizados
    \item Implementar monitoramento de integridade de arquivos (AIDE, Tripwire)
\end{enumerate}

\end{document}